\documentclass[a4paper]{article}
\usepackage[utf8]{inputenc}
\usepackage[spanish]{babel}
\usepackage{csquotes}
\usepackage{hyperref}
\usepackage[style=authoryear,backend=biber,citestyle=apa,maxnames=2]{biblatex}
\addbibresource{bib1.bib}
\nocite{*}


\title{Curso Introductorio a \LaTeX}
\author{ECEL Research Group}
\date{Abril 2021}

\begin{document}
\maketitle
\section*{Outline}
\textbf{Día 1}: Manejo de texto y formateo de un documento
\begin{enumerate}
    \item Introducción: 
    \begin{itemize}
       \item ¿Qué es \LaTeX? ¿Para qué sirve? ¿Que veremos en este curso?
    \item Breve historia
     \item Teclas de importancia
     \item Descarga de la distribución, IDEs\footnote{Se describirá como descargar una distribución de \TeX, tipos de builder, pero para todo el curso utilizaremos Overleaf para la facilidad del usuario.}
    \end{itemize}
    \item Empezando un documento
    \begin{itemize}
            \item El préambulo: clases de documentos, tamaño de letra predeterminado, paquetes \texttt{inputenc}, \texttt{babel}, etc. 
        \item Formatear el documento: márgenes, orientación, interlineado, etc.
        
            \end{itemize}
    \item Manejando texto
    \begin{itemize}
        \item Alineación del texto,  tamaño del texto, tipos de texto
        \item Listas numeradas y no numeradas, secciones y subsecciones. 
        \item Construyendo un título, carátulas
        \item  Encabezados, pies de página, índices, notas al pie
        \item Manipulando listas y numeración de índices
        \item Utilizando comillas y blockquotes, mas uso de paquetes. 
        \end{itemize}
 \item Miscelaneos 
 \begin{itemize}
              \item Tablas, Labels
              \item Integración con Microsoft Excel: rápidamente hacer tablas complicadas
                \item Cross-Referencing
                \item Hipervínculos
 \end{itemize}

\end{enumerate}
\textbf{Día 2}: Manejo de Notación Matemática
\begin{enumerate}
    \item Math mode
    \begin{itemize}
        \item Símbolos más básicos para el álgebra
        \item Subíndices y superíndices
        \item Letras griegas y funciones trigonométricas, uso de paquetes para algunos símbolos especiales (porcentajes)
        \item Paréntesis, corchetes, llaves, etc. 
         \end{itemize}
            \item Manejo de matemáticas avanzado
         \begin{itemize}
               \item Modos diferentes: alineación, numeración de ecuaciones, inclusión en tablas de contenidos
        \item Avanzados: integrales, diferenciales, matrices, sigma, econometría, \textit{display style}, etc.
        \item Programación de macros simples 
     \end{itemize}
\end{enumerate}
\textbf{Día 3:} Imágenes y Elaboración de Gráficos
\begin{enumerate}
    \item Imágenes
    \begin{itemize}
        \item Como poner una imagen, darle título, moverla, etc. 
        \item Float, formato tipo APA, etc. 
        \item Manejo de captions, labels
        \end{itemize}
 \item Elaborando gráficos con \textsf{tikz}/ \textsf{pgplots}
\begin{itemize}
     \item Introducción a los paquetes
 \item Realización de gráficos de dispersión con datos en archivos text (con \textsf{pgfplots}
 \item Dibujar funciones
 \item Otros gráficos
 
\end{itemize}
\item Una breve introducción a presentaciones con \LaTeX: Beamer
\end{enumerate}
\textbf{Día 4}: Manejando \LaTeX \ para un proyecto final o trabajo de titulación
\begin{enumerate}
    \item Utilizando Overleaf
\begin{itemize} 
    \item Como descargar proyectos, compilarlos sin internet
    \item Subir proyectos del computador al Overleaf
    \item Proyectos \textit{multi-file}
    \item Documentos colaborativos
\end{itemize}
\item Realizando bibliografías con el paquete \textsf{biblatex}
\begin{itemize}
    \item Código necesario en el preámbulo para \textsf{biblatex}
    \item Integración con Citavi para escribir los códigos de bibliografía
    \item Formateo de citas y bibliografía
    \end{itemize}
    \item Integración con otros programas
\begin{itemize}
    \item Integración con Rstudio: el paquete \texttt{stargazer} para exportación de tablas y estadísticas
    \item Integración con Stata: exportando output en formato \LaTeX
    \item Integración con Microsoft Word: escribiendo código \LaTeX \ en Word para matemáticas
    \item Mathpix
\end{itemize}
\item Consejos varios para \textit{debugging} del código y conclusión
\end{enumerate}
\printbibliography[title=Material de Referencia]
\end{document}
